%!TEX TS-program = xelatex 
%!TEX encoding = UTF-8 Unicode

%\documentclass[ignorenonframetext, t]{beamer}
\documentclass[a4paper]{article}
\usepackage{beamerarticle}
\usepackage{thucourse}
%\includeonlyframes{current}

\title{光阴极物理}
\subtitle{高亮度电子束物理及应用}
\author{唐传祥}
\institute{清华大学工程物理系}
\date{2016--2017 秋季学期}

\begin{document}

\begin{frame}
  \titlepage
\end{frame}

\begin{frame}
  \frametitle{目录}
  \begin{multicols}{2}
  \tableofcontents
  \end{multicols}
\end{frame}

\section{光电效应}

\subsection{光电效应基础知识}

\begin{frame}[label=current]
  \frametitle{光电效应的基本概念}
  \begin{concept}
  	\alert{光电效应}是指光束照射在金属表面会使其发射电子的物理现象,发射出的电子被称为\alert{光电子}。
	\end{concept}
	
	\only<1>{\begin{center}
	\includegraphics[width=0.5\linewidth]{photoelectric}
	\end{center}}
	
	\only<2>{
  \begin{itemize}
	  \item 对于特定金属而言,入射光频率必须大于某特定值才能产生光电效应,光电效应的发生与否与入射光强度没有关系
	  \item 光电效应是一种量子效应,无法用 Maxwell 体系的经典电磁理论解释,必须借助\alert{光量子}的概念才能解释
	  \item 光电效应可用下面著名的公式描述:
	  \begin{formula}{光电效应光电子最大能量}
	  \[
	  E = h\nu-\phi
	  \]
	  \scriptsize{其中$E$是出射光电子最大能量,$h\nu$为入射光子能量,$\phi$称为金属的逸出功或功函数(work function)}
	  \end{formula}
  \end{itemize}}
\end{frame}

\begin{frame}
  \frametitle{光电效应研究的里程碑}
  \begin{itemize}
	  \item 1887 年 H. Hertz 发现用紫外线照射金属电极会更容易产生电火花,当时被称为赫兹效应,这是实验上首次发现光电效应
	  \item 1899 年 J. Thomson 实验验证了电子的存在
	  \item 1902 年 P. Lenard 实验定性发现出射电子能量与入射光频率正相关,而与入射光强度无关,这与 Maxwell 提出的光的波动理论矛盾
		\item 1905 年 A. Einstein 发文提出光量子概念解释光电效应的实验现象
		\item 1914 年 R. Millikan 实验验证了 Einstein 的光电效应公式
  \end{itemize}
\end{frame}

\subsection{单光子/多光子光电效应}

\begin{frame}
	\frametitle{单光子光电效应}
	\begin{concept}
  	\alert{单光子光电效应}就是金属内电子吸收单个光子的光电效应。
	\end{concept}
	\only<1>{
	\begin{physics}
		金属内处于费米能级(Fermi Energy)以下的束缚态电子吸收单个光子跃迁到高能态,进而成为自由电子向四周运动;经过若干次散射(能量损失)后部分高能态电子会到达金属表面,其中又有部分电子具有足够能量,克服金属表面势垒后从金属表面逸出,成为光电子。
	\end{physics}}
	\only<2>{
  \begin{itemize}
	  \item 光电子数目(光电子产额)与入射光强度成正比,因此又被称为线性光电效应,可以定义光阴极的量子效率 QE 如下:
	  \begin{formula}{量子效率 QE}
	  \[
	  QE = \frac{N_e}{N_p} = \frac{Q/e}{IS\tau/h\nu} = \frac{J}{I}\cdot\frac{h\nu}{e}
	  \]
	  \scriptsize{其中$N_e$为出射光电子数,$N_p$为入射光子数,$Q$是出射束团电荷量,$J$是出射电流密度,$I$是入射光功率密度,$h\nu$为入射光子能量,$e$为电子电荷量,$S$为入射光横向面积,$\tau$为入射光脉宽}
	  \end{formula}
	  \item 入射光子能量必须高于金属的逸出功才可以发生单光子光电效应
  \end{itemize}}
\end{frame}

\begin{frame}
	\frametitle{多光子光电效应}
	\begin{concept}
  	\alert{多光子光电效应}就是金属内电子吸收多个光子的光电效应。
	\end{concept}
	\only<1>{
	\begin{physics}
		金属内处于费米能级以下的束缚态电子同步或非同步吸收多个光子跃迁到高能态,进而成为自由电子向四周运动;经过若干次散射(能量损失)后部分高能态电子会到达金属表面,其中又有部分电子具有足够能量,克服金属表面势垒后从金属表面逸出,成为光电子。
	\end{physics}}
	\only<2>{
  \begin{itemize}
	  \item 光电子产额与入射光强度成幂函数关系,因此又被称为高阶光电效应或非线性光电效应;N光子光电效应的光电流密度J与入射光强度I成N次幂关系:
	  \[
	  J \propto I^N
	  \]
	  \item 入射光子能量小于金属逸出功也可能发生多光子光电效应
	  \item 入射光照射金属时会同时发生多个高阶光电效应
	  \item 高阶光电效应的光电子产额随阶数的增大迅速衰减
  \end{itemize}}
\end{frame}

\subsection{零阶光电效应}

\begin{frame}
	\frametitle{热发射}
	\begin{concept}
  	\alert{热发射}即因金属表面电子热运动而产生的电子发射。
	\end{concept}
	\only<1>{
	\begin{physics}
		对于温度高于绝对零度的金属,假定其温度为 $T$($T > 0$),由于金属内部的电子能量服从 Fermi-Dirac 分布($T$ 较小时近似 Maxwell-Boltzmann 分布),其拖尾部分的电子处于较高能态,从而具有足够的能量克服金属表面势垒,逸出金属表面。
	\end{physics}
	\begin{formula}{热发射电流密度}
	\[
	J = AT^2\exp\left(-\frac{\phi}{kT}\right)
	\]
	\scriptsize{其中 $A$ 是Richardson常数,约为 \SI{1202}{\micro A.\milli m^{-2}.K^{-2}};$\phi$ 是金属逸出功,$k$ 是 Boltzmann 常数}
	\end{formula}}
	\only<2>{
  \begin{itemize}
	  \item 热发射是光电发射的特例,电子吸收 0 个光子而逸出金属表面,即 0 阶光电发射,电子逃逸是通过热运动完成的
	  \item 热发射电流只与金属温度$T$有关,与入射光强度大小无关
	  \item 金属温度 $T = 0$ 时,热发射电流 $J = 0$;$T > 0$ 时热发射电流 $J > 0$
	  \item 热发射电流的强度很小,在 $T = \SI{300}{K}$ 时热发射电流在 \SI{1}{\micro A} 量级
  \end{itemize}}
\end{frame}

\begin{frame}
	\frametitle{场致发射}
	\begin{concept}
  	\alert{场致发射}即因金属表面存在很强的引出电场而产生的电子发射。
	\end{concept}
	\only<1>{
	\begin{physics}
	当金属表面存在很强的引出电场(即电场线指向金属表面的电场)时,金属表面势垒会在电场的影响下变成准三角形,其厚度变小,从而金属表面的电子可以通过\alert{量子隧穿}效应穿越表面势垒,逸出金属表面。
	\end{physics}
	\begin{formula}{场致发射电流密度}
	\[
	J = \frac{A\beta^2 E^2}{\phi}\exp\left(-\frac{B\phi^{3/2}}{\beta E}\right)
	\]
	\scriptsize{其中 $E$ 是电场强度,$\beta$ 为场增强因子,$A, B$ 是与金属表面情况有关的参数,可通过实验拟合得到}
	\end{formula}}
	\only<2>{
  \begin{itemize}
	\item 场致发射是光电发射的特例,电子吸收 0 个光子而逸出金属表面,即 0 阶光电发射,电子逃逸是通过量子隧穿完成的
	\item 金属温度 $T = \SI{0}{K}$ 时,依然有场致发射电流 $J > 0$
	\item 场致发射电流对金属温度 $T$ 的变化不敏感
	\item 场致发射电流随引出电场的梯度下降而迅速减小,$E = 0$ 时场致发射电流 $J = 0$
  \end{itemize}}
\end{frame}

%\subsection{体光电效应与表面光电效应}
%
%\begin{frame}
%	\frametitle{体光电效应}
%\end{frame}
%
%\begin{frame}
%	\frametitle{表面光电效应}
%\end{frame}

\section{光阴极电子发射物理模型}

%\subsection{光阴极电子发射模型简述}
%
%\begin{frame}
%	\frametitle{模型演化}
%\end{frame}
%
%\begin{frame}
%	\frametitle{唯象理论与量子理论}
%\end{frame}

\subsection{金属中的自由电子}

\begin{frame}
	\frametitle{Drude-Sommerfeld 模型}
	\framesubtitle{模型概述}
	Drude-Sommerfeld 模型是一个用于解释固体宏观性质(导电、传热、比热等)的半经典模型,它结合了经典的 Drude 模型与量子力学中的部分观点(例如作用量量子化),虽然简单却取得了巨大成功。
	\begin{background}
	P. Drude 于 1900 年提出了 Drude 模型,该模型的要点为:
	\begin{itemize}
	\item 固体中的电子可看做经典的带电小球
	\item 固体中的原子核可视为极重的几乎静止的大球
	\item 固体中的电子不停地与原子核发生碰撞
	\end{itemize}
	该模型成功导出了固体中电子的运动方程以及固体中的电流密度与局域电场的线性关系,却很大地高估了固体的比热。其根本原因在于该模型的经典本质。A. Sommerfeld 于 1933 年将部分量子力学的观点应用于 Drude 模型,成功地解释了固体的比热,这就是 Drude-Sommerfeld 模型。
	\end{background}
\end{frame}

\begin{frame}
	\frametitle{Drude-Sommerfeld 模型}
	\framesubtitle{金属中电子的能态密度}
	\only<1>{Drude-Sommerfeld 模型中,固体中电子形态为自由电子费米气体(也称为简并电子气),自由电子费米气体的一个重要概念就是\alert{能态密度(Energy Density of States,EDOS)}。\\~\\
	金属中的电子的 EDOS 可以推导如下:}
	\only<-2>{\begin{derivation}
	\only<1>{
	固体中电子能量为:
	\[\varepsilon = \dfrac{\hbar^2k^2}{2m}\]
	固体中波的普遍态密度: 
	\[\rho(k) =\left(\frac{V}{2\pi^2}\right)k^2\]}
	\only<2>{
	考虑到单位体积与电子自旋简并数 2,EDOS 可表示为:
	\[g(\varepsilon) =\dfrac{2}{V}\rho(k)\dfrac{dk}{d\varepsilon}\]
	结合前面两式我们有:
	\[g(\varepsilon) = \frac{1}{2\pi^2}\left(\frac{2m}{\hbar^2}\right)^{3/2}\sqrt{\varepsilon}\]}
	\end{derivation}}
	\only<2>{
	于是我们得到:
	\begin{formula}{金属中电子的能态密度}
	\[g(\varepsilon) = \frac{1}{2\pi^2}\left(\frac{2m}{\hbar^2}\right)^{3/2}\sqrt{\varepsilon}\]
	\end{formula}}
	\only<3>{
	下图展示了金属中电子能态密度与能级的开方正比关系:\\~\\
	\begin{center}
	\includegraphics[width=0.5\linewidth]{EDOS}
	\end{center}}
\end{frame}

\begin{frame}
	\frametitle{Drude-Sommerfeld 模型}
	\framesubtitle{金属中电子的速度分布}
	\only<1>{
	量子力学指出,金属中的电子满足 Fermi-Dirac 分布:
	\begin{formula}{Fermi-Dirac 分布}
	\[f(\varepsilon) = \frac{1}{e^{(\varepsilon-\mu)/kT}+1}\]
	\scriptsize{这里 $\varepsilon$ 是能级,$\mu$ 被称为 Fermi 能量,$k$ 是 Boltzmann 常数,$T$ 是金属温度}
	\end{formula}
	Fermi-Dirac 分布描述了金属中的特定能级上的电子占有率。当金属温度 $T\rightarrow \SI{0}{K}$ 时,可以看出对于小于 $\mu$ 的能级电子占有率为 1,大于 $\mu$ 的能级电子占有率为 0。\\~\\
	利用金属中的电子能态密度公式以及 Fermi-Dirac 分布公式,可以得到金属中单位体积电子的速度分布 $n(u, v, w)$。}
	\only<2->{
	\begin{derivation}
	\only<2>{
	金属中单位体积电子的能量分布可写成 $n(\varepsilon)=g(\varepsilon)f(\varepsilon)$,那么有下面等式成立:
	\[\iiint\limits_{\frac{1}{2}m(u^2+v^2+w^2)\in [\varepsilon, \varepsilon+d\varepsilon]}\Zskip\Zskip n(u, v, w)dudvdw = n(\varepsilon)d\varepsilon\]
	假定动量空间均匀分布,那么 $n(u,v,w)=n^{*}\left(\sqrt{u^2+v^2+w^2}\right)$,将 $(u,v,w)$ 变换到球坐标中,积分便可得到:
	\[n(u, v, w) = \dfrac{m^{\frac{3}{2}}}{2\pi\cdot2^{\frac{3}{2}}}\cdot\frac{1}{\sqrt{\varepsilon}}\cdot n(\varepsilon)\]}
	\only<3>{
	代入 $n(\varepsilon)$ 并做变量代换,即得到:
	\[n(u, v, w) = 2\left(\frac{m}{h}\right)^3\frac{1}{e^{\left[\frac{1}{2}m\left(u^2+v^2+w^2\right)-\mu\right]/kT}+1}\]}
	\end{derivation}}
	\only<3>{
	因此我们得到:
	\begin{formula}{金属中单位体积电子的速度分布}
	\[n(u, v, w) = 2\left(\frac{m}{h}\right)^3\frac{1}{e^{\left[\frac{1}{2}m\left(u^2+v^2+w^2\right)-\mu\right]/kT}+1}\]
	\end{formula}}
\end{frame}

\subsection{阴极上的空间电荷限}

\begin{frame}
	\frametitle{Child-Langmuir 模型}
	\framesubtitle{模型概述}
	Child-Langmuir 模型是研究阴极上\alert{空间电荷限电流(Space-Charge Limited Current, SCLC)}的模型。
	\only<1>{\begin{center}
	\includegraphics[width=0.9\linewidth]{child}
	\end{center}}
	
	\only<2>{
	\begin{background}
	C. Child 于 1911 年提出 Child 定律,该定律是说在平行两极板间的电压 $V$ 固定的情况下,其极板间电流密度存在最大值 $J$,并且 $J$ 与 $V$ 成 3/2 次方关系,因此又被称为 3/2 次方定律。1911 年发表的情况针对的是离子流,I. Langmuir 在 1913 年将 Child 的方法应用于电子流,因此该定律又被称为 Child-Langmuir 定律。
	\end{background}
	\begin{physics}
	阴极上空间电荷限的本质是:在两极板间的电流产生的空间电荷场会减小发射极的引出电场,当电流大到一定程度时,发射极表面的总引出电场降到 0,逸出的电子无法获得加速,发射电流无法进一步增加,因此电压固定时发射电流存在饱和值。
	\end{physics}}
\end{frame}

\begin{frame}
	\frametitle{Child-Langmuir 模型}
	\framesubtitle{稳态情形}
	\only<1>{
	Child-Langmuir 模型的假定如下:
	\begin{itemize}
	\item 不考虑相对论效应
	\item 忽略极板间运动的粒子间的散射
	\item 极板间的粒子是同一的(同质量,同电荷量)
	\item 阴极上的初始发射速度为 0
	\end{itemize}
	那么当两极板间电流处于稳态时,我们可以直接利用 Poisson 方程导出空间电荷限电流公式:}
	\only<-3>{
	\begin{derivation}
	\only<1>{
	我们考虑两极板间电流中的一点,那么有 Poisson 方程成立:
	\[\nabla^2V=-\frac{\rho}{\epsilon_0}\]
	其中 $V$ 为该点电势,$\rho$ 为该点电荷密度。}
	\only<2>{
	由于阴极上($V=0$)粒子的初始动能为 0,根据能量守恒,电势 $V$ 处粒子的能量满足:
	\[\frac{1}{2}mv^2+qV=0\]
	考虑到电荷守恒及稳态条件,有:
	\[\nabla\cdot\vec{J}=-\frac{\partial \rho}{\partial t}= 0\]
	而电流密度$\vec{J}$满足:
	\[\vec{J}=\rho \vec{v}\]}
	\only<3>{
	我们仅考虑一维情形(束流沿 $z$ 轴运动),可知 $J$ 是个常量,联立以上各式可得关于电势 $V$ 的方程:
	\[\frac{d^2V}{dz^2}+AV^{-\frac{1}{2}}=0\]
	其中 $A=-\frac{J}{\varepsilon_0}\sqrt{-m/2q}$。解上面方程并代入边界条件:
	\begin{eqnarray*}
	&&V|_{z=0}=0\\
	&&V|_{z=d}=V_d
	\end{eqnarray*}
	可得电压$V_d$与最大电流密度 $J$ 之间的关系为:
	\[J = \frac{4\varepsilon_0}{9}\sqrt{-2q/m}\cdot \frac{V_d^{\frac{3}{2}}}{d^2}\]}
	\end{derivation}}
	\only<4>{
	对于电子而言,我们就得到阴极上的空间电荷限电流密度为($V_d\rightarrow V$):
	\begin{formula}{阴极的空间电荷限电流密度}
	\[J = \frac{4\varepsilon_0}{9}\sqrt{2e/m}\cdot \frac{V^{\frac{3}{2}}}{d^2}\]
	\end{formula}
	注意上式我们是在平行电极板情形下推导得到的,但是该公式对于圆柱电极板情形也成立,大家可以自己推导一下作为练习。
	\\~\\
	在带有电子枪的电子器件中,人们定义\alert{导流系数 $P=I/V^{3/2}$} 来衡量电子枪的发射能力,$P$ 的单位为朴。一般微波器件的电子枪,导流系数在 \SIrange[range-phrase = --]{0.1}{1}{\micro P} 的范围。导流系数反映了阴极发射电流I与阴阳极间电压V之间的关系,它仅取决于枪的几何尺寸。}
\end{frame}

%\begin{frame}
%	\frametitle{Child-Langmuir 模型}
%	\framesubtitle{非稳态情形}
%\end{frame}

\subsection{光阴极的热发射电流}

\begin{frame}
	\frametitle{Richardson-Dushman 模型}
	\framesubtitle{模型概述}
	Richardson-Dushman 模型是用来研究金属表面热发射现象的模型。采用简并电子气的观点,金属表面附近的电子动能服从 FD 分布,若金属表面温度不为 \SI{0}{K},则存在较高能态的电子,当其垂直金属表面方向的动能足够大,就有可能克服表面势垒逸出金属表面,形成热发射电流。
	\only<1>{\begin{center}
	\includegraphics[width=0.5\linewidth]{richardson}
	\end{center}}
	
	\only<2>{
	\begin{background}
	1897 年 J. Thomson 发现电子后,O. Richardson 开始研究热电流现象。1901 年 O. Richardson 发表了他对金属丝上热电流的实验观测结果,发现金属丝上热电流的大小与金属丝的温度成指数关系;1923 年 S. Dushman 得到了热发射电流密度的现代形式,也即 Richardson-Dushman 公式。
	\end{background}}
\end{frame}

\begin{frame}
	\frametitle{Richardson-Dushman 模型}
	\framesubtitle{热发射电流密度}
	\only<1-2>{
	\only<1>{
	阴极上的热发射电流密度可以利用 Drude-Sommerfeld 模型中的电子速度分布公式做近似直接导出。}
	\begin{derivation}
	\only<1>{
	金属表面的电子速度分布遵守:
	\[n(v_x, v_y, v_z) = 2\left(\frac{m}{h}\right)^3\frac{1}{e^{(E-E_F)/kT}+1}\]
	由于只有动能 $E$ 远高于费米能级 $E_F$ 的电子才能克服金属表面势垒 $U$ 逸出,我们统计热电流时可将 Fermi-Dirac 分布近似为 Maxwell-Boltzmann 分布:
	\[\frac{1}{e^{(E-E_F)/kT}+1} \rightarrow e^{-(E-E_F)/kT}\]}
	\only<2>{
	假定阴极法向沿 $+x$ 方向,那么热发射电流密度 $J$ 可写成:
	\[J = \iiint\limits_{mv_x^2/2 > U}\zskip ev_xn(v_x, v_y, v_z)dv_xdv_ydv_z\]
	将近似后的电子速度分布公式代入,利用高斯积分:
	\[\int_{-\infty}^{\infty}e^{-\alpha z^2}dz=\sqrt{\frac{\pi}{\alpha}}\]
	可以得到:
	\[J = \frac{4\pi emk^2}{h^3}T^2\exp\left(\frac{E_F-U}{kT}\right)\]}
	\end{derivation}}
	\only<3>{
	由此我们得到热发射电流密度公式:
	\begin{formula}{Richardson-Dushman 公式}
	\[
	J = AT^2\exp\left(-\frac{\phi}{kT}\right)
	\]
	\scriptsize{其中 $A=4\pi emk^2/h^3\approx \SI{1202}{\micro A.\milli m^{-2}.K^{-2}}$,称为Richardson常数;$\phi=U-E_F$,定义为金属的逸出功}
	\end{formula}
	Richardson-Dushman 公式最早被用于实验测量钨和其他金属的逸出功,这也是测量逸出功最古老的方法。现代一般采用更精确的光电发射的方法来测量逸出功。}
\end{frame}

\subsection{光阴极的暗电流}

\begin{frame}
	\frametitle{Fowler-Nordheim 模型}
	\framesubtitle{模型概述}
	Fowler-Nordheim 模型是用来研究固体表面在高引出电场作用下的量子隧穿(即场致发射)电流的模型。
	\only<1>{\begin{center}
	\includegraphics[width=0.6\linewidth]{tunnel}
	\end{center}}
	
	\only<2>{
	\begin{background}
%	场致发射现象最早可能是 J. Winkler 于 1744 年在研究金属丝阴极时发现:他发现金属丝阴极会自行放电。
	1897 年 J. Thomson 发现电子后,热发射和光电发射现象先后被发现,人们开始理解能量足够克服金属表面的势垒的电子可以从金属内部发射出来。1913 年,物理学家们开始怀疑金属表面的电子发射中有除了热发射和光电发射现象之外的新的物理现象存在---也就是场致发射现象。1922 年 J. Lilienfeld 发表了第一篇明确地阐述场致发射现象的实验文章。同年,R. Millikan 等人开始对该现象感兴趣并做了实验研究,他们首先画出 $i-V$ 图并试图用直线拟合,然而这种尝试并不成功;随后人们尝试了 $\log(i)-V$ 图及 $\log(i)-V^{1/2}$ 图,依然没有得到线性关系。首次突破在 1928 年出现,C. Lauritsen 发现场致发射的 $\log(i)-1/V$ 图是线性的,J. Oppenheimer 提出了这种发射来源于量子隧穿的观点,随后 R. Fowler 和 L. Nordheim 给出了场致发射电流的解析公式,这就是著名的 Fowler-Nordheim 公式。
	\end{background}}
\end{frame}

\begin{frame}
	\frametitle{Fowler-Nordheim 模型}
	\framesubtitle{隧穿概率}
	\only<1>{
	在光阴极领域,阴极表面场致发射产生的电流又被称为\alert{暗电流}。我们下面就尝试推导暗电流密度的解析公式。
	\begin{derivation}
	考虑阴极表面附近的电子,假定阴极法向沿 $+x$ 方向,利用简并电子气的电子速度分布 $n(v_x, v_y, v_z)$,可写出场致发射电流密度为:
	\[
	J = \iiint\limits_{v_x \in [0, +\infty]}\zskip ev_xn(v_x, v_y, v_z)\cdot \Theta(v_x)dv_xdv_ydv_z
	\]
	其中 $\Theta(v_x)$ 为 $x$ 方向速度为 $v_x$ 的电子的隧穿概率。
	\end{derivation}
	因此想计算场致发射电流密度,需计算电子隧穿概率 $\Theta(v_x)$。}
	\only<2>{我们采用 \alert{WKB 近似(WKB (Wigner, Kramers, Brillouin) approximation)}来计算量子隧穿概率。}
	\only<2-4>{
	\begin{derivation}
	\only<2>{
	假定阴极法向沿 $+x$ 方向,我们计算阴极表面势垒分布为 $V(x)$ 情形下的电子波函数,从而得到隧穿概率。设此时电子波函数为 $\Phi(x)$,利用一维含时 Schr\"odinger 方程,我们有:
	\[\frac{d^2\Psi}{dx^2}=\frac{2m(V-E_x)}{\hbar^2}\Psi\]
	其中 $E_x$ 为电子在 $x$ 方向的动能。考虑在 $[x, x+dx]$ 一段的势垒 V,我们近似认为其为常数,那么可解出波函数 $\Phi$ 在此区间满足:
	\[\Phi(x+dx)=\Phi(x)\exp\left(-\frac{\sqrt{2m[V(x)-E_x]}}{\hbar}dx\right)\]}
	\only<3>{
	于是 $\Psi$ 在整个 $x$ 正半轴上的可写为:
	\[\Phi(x)=\Phi(0)\exp\left(-\int_0^x\frac{\sqrt{2m[V(s)-E_x]}}{\hbar}ds\right)\]
	上面的公式就称为 \alert{WKB 近似}。有了波函数,我们便可以计算特定形状势垒下的隧穿概率:
	\[\Theta(E_x)=\frac{\Psi(L)\Psi^*(L)}{\Psi(0)\Psi^*(0)}=\exp\left(-\frac{2\sqrt{2m}}{\hbar}\int_0^L\zskip\sqrt{V(s)-E_x}ds\right)\]
	式中 $L$ 为势垒宽度,也即相对(能量 $E_x$ 的)势垒高度刚好等于 0 的位置。下面以外场为恒定电场且不考虑空间电荷力时的三角(ET,Exact Triangular)势垒为例计算隧穿概率。}
	\only<4>{
	假定外场梯度为 $F$,那么 $V(x)-E_x=\phi-eFx$,其中 $\phi$ 为相对电子 $x$ 方向动能 $E_x$ 的势垒高度,此时势垒宽度 $L=\phi/eF$。代入 $\Theta$ 的计算公式,就得到:}
	\end{derivation}}
	\only<4>{
	\begin{formula}{ET 势垒的隧穿概率}
	\[\Theta(E_x)=\exp\left(-\frac{4\sqrt{2m}}{3e\hbar}\frac{\phi(E_x)^{3/2}}{F}\right)\]
	\scriptsize{
	其中 $\phi(E_x)=\phi_w+E_F-E_x$,是相对于 $x$ 方向动能为 $E_x$ 的电子的金属表面势垒高度}
	\end{formula}
	有了隧穿概率,我们可以代回到场致发射电流密度公式中,计算暗电流密度。}
\end{frame}

\begin{frame}
	\frametitle{Fowler-Nordheim 模型}
	\framesubtitle{暗电流密度}
	\only<1-2>{
	\begin{derivation}
	\only<1>{
	容易将场致发射电流密度公式化为下面形式:
	\[J = A\cdot\frac{T}{k}\int_0^{\infty} dE_x\Theta(E_x)\ln\left(e^{-\frac{E_x-E_F}{kT}}+1\right)\]
	其中 $A$ 为 Richardson 常数。考虑到场致发射电子大部分能量在 Fermi 能级附近,我们可以将一般形式的 $\Theta(E_x)$ 在 $E_F$ 附近做 Taylor 展开到二阶,有:
	\[\Theta(E_x)=\Theta(E_F)\cdot e^{\left(\frac{2\sqrt{2m}}{\hbar}(E_x-E_F)\int\limits_0^L\frac{ds}{\sqrt{V(s)-E_F}}\right)}\]
	将上式代入 $J$ 的表达式,考虑到场致发射电流主要由 $E_x$ 在 Fermi 能级附近的电子贡献,可以将 $J$ 中的积分区间扩充至 $(-\infty, \infty)$。}
	\only<2>{
	通过变量代换及积分公式:
	\[\int_0^{\infty}du u^{-\alpha-1}\ln(u+1)=\frac{\pi}{\alpha\sin(\alpha\pi)},\quad\mathrm{Re}(\alpha)\in(0, 1)\]
	可得到:
	\[J = AT^2\frac{\Theta(E_F)}{C^2\mathrm{Sa}(C\pi)}\]
	其中 $C$ 满足:
	\[C = kT\frac{\sqrt{2m}}{\hbar}\int_0^L\frac{ds}{\sqrt{V(s)-E_F}}\]
	代入 ET 势垒的隧穿概率的相关计算结果,最终我们有:}
	\end{derivation}}
	\only<3>{
	\begin{formula}{场致发射电流密度}
	\[J=\frac{AF^2}{\phi_w}\exp\left(-\frac{B\phi_w^{3/2}}{F}\right)\frac{1}{\mathrm{Sa}(CT\phi_w^{1/2}/F)}\]
	\scriptsize{其中 $A=e^3/8\pi h$,$B=4\sqrt{2m}/3e\hbar$,$C=2\pi k\sqrt{2m}/e\hbar$}
%	\scriptsize{其中 $A=e^3/8\pi h\approx\SI{1.541434e-6}{A.eV.V^{-2}}$,$B=4\sqrt{2m}/3e\hbar\approx\SI{6.830890}{eV^{-3/2}.V.\nano m^{-1}}$,$C=2\pi k\sqrt{2m}/e\hbar\approx\SI{2.773902}{K^{-1}.eV^{-1/2}.V.\micro m^{-1}}$
	\end{formula}
	当温度 $T\rightarrow \SI{0}{K}$ 时,$\mathrm{Sa}$ 函数趋近于 1,由此:
	\begin{formula}{Fowler-Nordheim 公式}
	\[J=\frac{AF^2}{\phi_w}\exp\left(-\frac{B\phi_w^{3/2}}{F}\right)\]
	\scriptsize{其中 $A=e^3/8\pi h\approx\SI{1.541e-6}{A.eV.V^{-2}}$,$B=4\sqrt{2m}/3e\hbar\approx\SI{6.831}{eV^{-3/2}.V.\nano m^{-1}}$,$F$ 是金属表面场梯度}
	\end{formula}}
\end{frame}

\subsection{光阴极的光电流}

\begin{frame}
	\frametitle{Fowler-Dubridge 模型}
	\framesubtitle{单光子光电发射}
	\only<1>{
	Fowler-Dubridge 模型研究了光电发射过程中光电流的属性。
	\begin{background}
	1931 年 R. Fowler 研究了光电发射中光电流随入射光子能量变化曲线的拖尾现象。他采用简并电子气的观点,并假设:
	\begin{itemize}
	\item 吸收光子后电子的能量略高于克服表面势垒所需能量,因此\alert{忽略电子间散射作用}
	\item 所有\alert{沿阴极法向的动能与吸收的光子能量之和}大于克服阴极表面势垒所需能量的电子都可从表面逃逸,这种电子叫做可出射电子
	\item 光电流正比于可出射电子数
	\end{itemize}
	根据以上假设,R. Fowler 获得了光电发射电流 $I$ 的表达式,成功地揭示了拖尾现象的机理。1933 年 L. Dubridge 扩充了 Fowler 的理论,计算了光电发射中光电子的能谱,因此该理论被称为 Fowler-Dubridge 理论。
	\end{background}}
	\only<2>{我们只演示 Fowler-Dubridge 模型中光电流公式的推导过程。}
	\only<2-3>
	{\begin{derivation}
	\only<2>{
	假定阴极表面法向沿 $+x$ 方向,利用简并电子气的速度分布公式,可以得到电子密度随 $v_x$ 的分布如下:
	\[
	n(v_x) = \frac{4\pi kT}{m}\left(\frac{m}{h}\right)^2\log\{1+e^{(\mu-\frac{1}{2}mv_x^2)/kT}\}
	\]
	根据 Fowler 的第二条假设,单位体积内所有可出射的电子数为:
	\begin{eqnarray*}
	&N &= \int_{\frac{1}{2}mv_x^2=\phi_w+\mu-h\nu}^{\infty}n(v_x)dv_x\\
	&&= \frac{2\pi kT}{m}\left(\frac{2kT}{m}\right)^{1/2}\left(\frac{m}{h}\right)^3\int_0^{\infty}\frac{\log\{1+e^{-y+(h\nu-\phi_w)/kT}\}}{\{y+(\mu+\phi_w-h\nu)/kT\}^{1/2}}dy
	\end{eqnarray*}}
	\only<3>{
	根据 Fowler 的第一条假设,上式积分分母中的 $y$ 相比于第二项可以忽略,所以上式可简化为:
	\[
	N = \frac{2\sqrt{2}\pi m^{3/2}}{h^3}\frac{(kT)^2}{(\mu+\phi_w-h\nu)^{1/2}}\int_0^{\infty}\log\{1+e^{-y+(h\nu-\phi_w)/kT}\}dy
	\]
	令 $F(x) = \int_0^{\infty}\log(1+e^{-y+x})dy$,则:
	\[
	N = \frac{2\sqrt{2}\pi m^{3/2}}{h^3}\frac{(kT)^2}{(\mu+\phi_w-h\nu)^{1/2}}F\left(\frac{h\nu-\phi_w}{kT}\right)
	\]
	根据 Fowler 的第三条假设,光电流 $I$ 正比于可出射电子数 $N$,我们就得到:
	\[
	I = CT^2F\left(\frac{h\nu-\phi_w}{kT}\right)
	\]
	}
	\end{derivation}}
	\only<4>{
	\begin{formula}{单光子发射光电流}
	\[
	I = CT^2F\left(\frac{h\nu-\phi_w}{kT}\right)
	\]
	\scriptsize{
	其中 $C$ 为常数,$T$ 为电子气温度,$F(x)$ 为 Fowler 函数,可以展成下面级数:
	\[
	F(x) =
	\begin{cases}
	e^x-\dfrac{e^{2x}}{2^2}+\dfrac{e^{3x}}{3^2}-\cdots & x \le 0\\[10pt]
	\dfrac{\pi^2}{6}+\dfrac{x^2}{2}-\left[e^{-x}-\dfrac{e^{-2x}}{2^2}+\dfrac{e^{-3x}}{3^2}-\cdots\right] & x > 0
	\end{cases}
	\]}
	\end{formula}
	Fowler 函数的特征是在 $x < 0$ 时,其值随 $x$ 的减小而迅速衰减。这个特性意味着波段合适的激光可以作为阴极的发射开关,即光电发射具有良好的可控性。
	}
\end{frame}

\begin{frame}
	\frametitle{Fowler-Dubridge 模型}
	\framesubtitle{多光子光电发射}
	\only<1>{
	J. Bechtel 于 1973 年在他的博士论文里推广了 Fowler-Dubridge 模型,将多光子发射的情形包含进来,他得到:
	\begin{formula}{多光子发射光电流密度}
	\[
	J_n = \alpha_nA\left(\frac{e}{h\nu}\right)^n(1-R)^n\cdot I^nT^2F\left(\frac{nh\nu-\phi_w}{kT}\right)
	\]
	\scriptsize{其中 $A$ 为 Richardson 常数 \SI{120}{A.cm^{-2}.K^{-2}},$R$ 为阴极表面反射率,$I$ 为入射激光功率密度,$\alpha_n$ 是与 n-光子电离常数成正比的常数。}
	\end{formula}
	取 $n=0$,此时 $\alpha_0 = 1$,我们再次得到热发射电流密度公式;取 $n=1$,我们就得到单光子光电流密度公式。}
	\only<2>{
	于是总发射电流密度为:
	\begin{formula}{总阴极发射电流密度}
	\[
	J = \sum_{i = 0}^{\infty}J_i
	\]
	\end{formula}
	此公式包含了热发射电流,线性光电流和高阶光电流。}
\end{frame}

\subsection{光阴极的量子效率和初始发射度}

\begin{frame}
	\frametitle{Spicer-Dowell 模型}
	\framesubtitle{三步模型}
	\only<-2>{
	光电子发射的三步模型:
	\begin{itemize}
	\item 光子\alert{激发(Excitation)}阴极内电子
	\item 电子\alert{输运(Transport)}到阴极表面
	\item 电子\alert{逃逸(Escape)}到真空
	\end{itemize}
	\only<1>{\begin{center}
	\includegraphics[width=0.6\linewidth]{3step}
	\end{center}}
	
	\only<2>{
	\begin{background}
	Fowler-Dubridge 模型仅仅考虑了阴极中电子在吸收光子后的运动状态,而未考虑光子吸收和电子输运的过程,因此无法用于计算光电子束较细致的参数,例如 QE 和发射度。1958 年 W. Spicer 开发了三步模型,考虑光子吸收和电子输运的过程,成功计算了 QE;2006 年 D. Dowell 利用三步模型,进行了简化并给出了 QE 和发射度的非常简洁的公式,并被实验所验证。
	\end{background}}}
	\only<3>{
	第一步:光子激发阴极内电子
	\begin{derivation}
	假设能量为 $h\nu$ 的光子垂直入射光滑金属阴极(沿 $-z$ 方向入射),由于光子被金属中电子吸收是一个泊松过程,因此光子被吸收的概率会随入射深度 $s$ 呈指数衰减,也即:
	\[
	p(\gamma: \mathrm{surface} \to s) = \frac{1}{\lambda_p}e^{-s/\lambda_p}
	\]
	其中 $p(\gamma: \mathrm{surface} \to s)$ 是进入阴极表面的光子在入射深度 $s$ 处被吸收的概率密度,$\lambda_p$ 是该能量光子在金属中的平均自由程。而能量为 $E$ 的电子吸收该光子,从而从能级 $E$ 跃迁到 $E+h\nu$ 的概率正比于能级 $E$ 的电子数与能级 $E+h\nu$ 的空位数的乘积:
	\[
	p(e^{-}: E\to E+h\nu) \propto f_{\mathrm{FD}}(E)g(E)\cdot [1-f_{\mathrm{FD}}(E+h\nu)]g(E+h\nu)
	\]
	其中 $f_{\mathrm{FD}}(E)$ 为 Fermi-Dirac 分布,$g(E)$ 是能级 $E$ 处的能态密度。
	\end{derivation}}
	\only<4>{
	第二步:电子输运到阴极表面
	\begin{derivation}
	被激发到高能态 $E+h\nu$ 的电子开始输运过程。假设被激发的电子运动方向随机,并在 $4\pi$ 方向角均匀分布,再限定被激发的电子能量仅略高于有效逸出功,则电子输运到阴极表面的概率等于电子朝向阴极表面运动并输运过程中无散射的概率(一旦发生散射电子丢失能量,剩余能量不足以克服有效势垒),也即:
	\[
	p(e^{-}: (E+h\nu, s, \theta) \to \mathrm{surface}) = e^{-s/(\lambda_{e-e}(E+h\nu)\cos\theta)}
	\]
	其中 $\lambda_{e-e}(E+h\nu)$ 是能量为 $E+h\nu$ 电子的散射平均自由程,$\theta$ 为电子运动方向与阴极表面法向($+z$)的夹角。
	\end{derivation}}
	\only<5>{
	第三步:电子逃逸到真空
	\begin{derivation}
	能量为 $E+h\nu$ 的电子以 $\theta$ 角传输至阴极表面时会尝试克服表面势垒而逃逸。假设所有 $+z$ 方向动能大于等于有效势垒高度(有效逸出功与 Fermi 能量之和)的电子都可以逃逸,可得逃逸概率为:
	\[
	p(e^{-}: \mathrm{escape}) = u[(E+h\nu)\cos^2\theta-\mu-\phi_{\mathrm{eff}}]
	\]
	其中 $u(x)$ 为阶跃函数:$x\ge 0$ 时 $u(x)=1$;$x<0$ 时 $u(x)=0$,$\mu$ 为 Fermi 能量。
	\end{derivation}}
\end{frame}

\begin{frame}
	\frametitle{Spicer-Dowell 模型}
	\framesubtitle{Schottky 效应和有效逸出功}
	\only<1>{
	\begin{concept}
	电子在逃逸出金属阴极表面时要克服表面势垒(真空势垒和镜像电荷势的总和)的作用;阴极表面的高引出场强会降低表面势垒的高度,从而降低电子所感受到的逸出功,这种效应叫做 \alert{Schottky 效应},电子所感受的逸出功叫做\alert{有效逸出功}。
	\end{concept}
	
	\begin{center}
	\includegraphics[width=0.7\linewidth]{schottky}
	\end{center}}
	\only<2>{
	\begin{derivation}
	对于单个电子而言,在金属阴极表面距离 $x$ 的位置,其电势能为:
	\[
	\Phi = \phi_{\mathrm{w}} - \frac{e^2}{16\pi\varepsilon_0x} - eE_0x
	\]
	其中 $\phi_{\mathrm{w}}$ 为金属的逸出功,$-e^2/16\pi\varepsilon_0x$ 是镜像电荷势能,$- eE_0x$ 是引出场势能。
	容易看出总势能在距离 $x$ 满足:
	\[
	x = \sqrt{\frac{e}{16\pi\varepsilon_0E_0}}
	\]
	时取得最大值,最大值即为有效逸出功:
	\end{derivation}
	\begin{formula}{考虑了 Schottky 效应的有效逸出功}
	\[
	\phi_{\mathrm{eff}} = \phi_{\mathrm{w}} - e\sqrt{\frac{eE_0}{4\pi\varepsilon_0}}
	\]
	\end{formula}}
\end{frame}

\begin{frame}
	\frametitle{Spicer-Dowell 模型}
	\framesubtitle{光阴极的量子效率}
	利用三步模型,可以计算光阴极的量子效率。
	\begin{derivation}
	\[\cdots\]
	\end{derivation}
	\begin{formula}{光阴极的量子效率}
	\[
	\mathrm{QE}(\omega) \approx\dfrac{1-R(\omega)}{1+\dfrac{\lambda_{\mathrm{opt}}}{\bar{\lambda}_{e-e}(\omega)}}\frac{(\hbar\omega-\phi_{\mathrm{eff}})^2}{8\phi_{\mathrm{eff}}(E_F+\phi_{\mathrm{eff}})}
	\]
	\end{formula}
\end{frame}

\begin{frame}
	\frametitle{Spicer-Dowell 模型}
	\framesubtitle{光阴极的初始发射度}
	\only<1>{利用三步模型,可以得到光阴极的初始发射度:
	
	\begin{center}
	\includegraphics[width=0.6\linewidth]{emittance}
	\end{center}}
	
	\only<2>{
	\begin{derivation}
	\[\cdots\]
	\end{derivation}
	\begin{formula}{光阴极的初始发射度}
	\[
	\varepsilon_{n} =\sigma\sqrt{\dfrac{\hbar\omega-\phi_{\mathrm{eff}}}{3mc^2}}
	\]
	\end{formula}}
\end{frame}

\begin{frame}
	\frametitle{Spicer-Dowell 模型}
	\framesubtitle{光阴极的折射定律}
	\only<1>{利用三步模型,可以得到光阴极的折射定律:
	
	\begin{center}
	\includegraphics[width=0.6\linewidth]{refraction}
	\end{center}}
	
	\only<2>{
	\begin{derivation}
	\[\cdots\]
	\end{derivation}
	\begin{formula}{光阴极表面的折射定律}
	\[
	\frac{\sin\theta_\mathrm{out}}{\sin\theta_\mathrm{in}}=\sqrt{\frac{E+h\nu}{E+h\nu-\mu-\phi_\mathrm{eff}}}
	\]
	\end{formula}}
\end{frame}

%\subsection{几个量子模型}
%
%\begin{frame}
%	\frametitle{Kantorovich-Keldysh 模型}
%\end{frame}
%
%\begin{frame}
%	\frametitle{Wan 模型}
%\end{frame}
%
%\subsection{粗糙表面光阴极上的电子发射}
%
%\begin{frame}
%	\frametitle{2D 正弦表面模型}
%\end{frame}
%
%\begin{frame}
%	\frametitle{* 3D 任意表面模型}
%\end{frame}
%
%\begin{frame}
%	\frametitle{* 纳米精细度模型} 
%\end{frame}

\section{几种典型的光阴极}

\subsection{传统光阴极}

\begin{frame}[label=current]
	\frametitle{金属光阴极}
	金属是常见的光阴极材料,其中 Cu 阴极和 Mg 阴极最为常用,LCLS 的光阴极注入器中使用的阴极就是 Cu 阴极。\\~\\
	
	\only<1>{\begin{center}
	\includegraphics[width=0.4\linewidth]{metal}
	\end{center}}
	
	\only<2>{金属阴极的特点如下:
	\begin{itemize}
	\item 量子效率 QE 较低,一般在 \SIrange[range-phrase = $\sim$]{1e-4}{1e-5}{}
	\alert{\item 飞秒(fs)级响应时间,可用于产生超短电子束团}
%	\item 归一化发射度通常在 \SIrange[range-phrase = $\sim$]{0.4}{1.0}{\milli m\cdot\milli rad/\milli m} 之间
	\item 对工作环境真空度要求低,一般 \SI{1e-8}{Torr} 即可工作
	\item 逸出功通常在 \SIrange[range-phrase = $\sim$]{3.5}{5}{eV} 范围,因此只能采用紫外(UV)激光(常用绿光倍频到 \SI{266}{\nano m})照射产生光电子
%	\item 性质稳定,可以在大气环境下短暂存放,方便加工与运输
%	\item 工作寿命长,大都在 \SI{1000}{h} 以上
	\end{itemize}}
\end{frame}

\begin{frame}[label=current]
	\frametitle{半导体光阴极}
	\framesubtitle{PEA 光阴极}
	半导体光阴极也是近年来使用较广的阴极,按照半导体的电子亲和势可分为正电子亲和势(PEA)阴极及负电子亲和势(NEA)阴极。常见的 PEA 阴极主要是碱金属锑化物及碱金属碲化物,如 $\sf Cs_2Te,Cs_3Sb,K_3Sb,Na_3Sb,Li_3Sb$,近年来多碱金属锑化物阴极也逐渐流行起来,例如 $\sf Na_2KSb,K_2CsSb$ 等。FLASH 所使用的光阴极就是 $\sf Cs_2Te$ 阴极。\\~\\
	
	\only<1>{\begin{center}
	\includegraphics[width=0.3\linewidth]{PEA}
	\end{center}}
	
	\only<2>{PEA 光阴极的特点如下:
	\begin{itemize}
	\item 量子效率 QE 较高,一般大于 5\%
	\item 皮秒(ps)级响应时间
	\item 对工作环境真空度要求相对较低,\SI{1e-9}{Torr} 即可工作
	\item 需要紫外(UV)激光驱动
	\end{itemize}}
\end{frame}

\begin{frame}[label=current]
	\frametitle{半导体光阴极}
	\framesubtitle{NEA 光阴极}
	常见的 NEA 阴极主要是砷化镓 $\sf GaAs$ 及氮化镓 $\sf GaN$ 阴极,JLab 的能量回收加速器 ERL 中使用的就是 $\sf GaAs$ 阴极。\\~\\
	
	\only<1>{\begin{center}
	\includegraphics[width=0.3\linewidth]{NEA}
	\end{center}}
	
	\only<2>{NEA 光阴极的特点如下:
	\begin{itemize}
	\alert{\item 量子效率 QE 很高,可达 10\%}
	\item 响应时间很长,一般在几十皮秒(ps)量级
	\item 对工作环境真空度要求很高,低于 \SI{1e-10}{Torr} 的真空度下才可以正常工作,且需要低温环境
	\item 近红外(NIR)激光即可驱动
	\alert{\item 可产生极化电子束}
	\end{itemize}}
\end{frame}

\subsection{几种新型光阴极}

\begin{frame}[label=current]
	\frametitle{表面等离激元光阴极}
	\framesubtitle{表面等离激元}
	\only<1-2>{Maxwell 电磁理论预言在两种介电常数实部反号的介质的交界面存在一种沿交界面传播的电磁-电荷密度振荡波,也称做\alert{表面等离激元(Surface Plasmon Polaritons, SPPs)}。利用经典的 Drude 模型可导出金属的介电常数实部为负,因此金属与介质交界面可以存在 SPPs。\\~\\
	
	\only<1>{\begin{center}
	\includegraphics[width=0.9\linewidth]{sps}
	\end{center}}
	
	\only<2>{表面等离激元具有下面特点:
	\begin{itemize}
	\item SPPs 存在一个频率上限($\omega_P/\sqrt{2}$),低于此频率的光才能激发 SPPs
	\item 入射光必须设法增加横向动量(横向波数)才能激发 SPPs
	\item SPPs 沿交界面传播,垂直于交界面衰减
	\item 特定波长和角度入射的光可被 SPPs \alert{完全共振吸收},此时交界面处会形成\alert{极高电场}
	\end{itemize}}}
	\only<3>{由于 SPPs 有提高入射光吸收率/增强阴极表面光场的性质,可以利用 SPPs 来加强光电发射,提高阴极的量子效率。在金属阴极表面激发起 SPPs,一般采用下面两种方式:
	\begin{itemize}
	\item 利用金属表面的光栅结构来提高入射光的横向波数,这种方式对应 \alert{NPC(Nano-Patterned Cathode)阴极}
	\item 通过将光射入电介质以提高入射光的横向波数,这种方式对应 \alert{ATR(Attenuated Total Reflectance)阴极}
	\end{itemize}
	\begin{center}
	\includegraphics[width=0.6\linewidth]{excitation}
	\end{center}}
\end{frame}

\begin{frame}[label=current]
	\frametitle{表面等离激元光阴极}
	\framesubtitle{NPC 光阴极}
	在金属表面刻上百纳米(nm)级的周期性结构,就构成了 NPC 光阴极。
	\begin{physics}
	激光入射到该纳米结构(对于激光的作用是反射光栅)时,其横向动量增加,能在表面激发起 SPPs,达到强化光电发射的目的。
	\end{physics}
	
	\only<1>{\begin{center}
	\includegraphics[width=0.3\linewidth]{NPC}
	\end{center}}
	
	\only<2>{NPC 光阴极具有下面特点:
	\begin{itemize}
	\item 量子效率可以比同材料光滑阴极高 2 $\sim$ 3 个量级
	\item 由于表面有周期性纳米结构,发射度一般比同材料光滑阴极更大
	\item 多采用多光子光电发射,尽可能利用 SPPs 对表面上激光功率密度的增强
	\item 可直接应用于微波电子枪中
	\end{itemize}}
\end{frame}

\begin{frame}[label=current]
	\frametitle{表面等离激元光阴极}
	\framesubtitle{ATR 光阴极}
	在透明电介质上镀一层几十纳米(nm)厚的金属薄膜,就构成了 ATR 光阴极。
	\begin{physics}
	激光从透明电介质端射入,其横向动量增加,因而能在金属与真空的交界面激发起 SPPs,达到强化光电发射的目的。与一般光阴极不同,激光并非是正入射到金属表面,打出反射电子形成光电流,而是背入射到金属表面,其激发的光电子需穿过几十纳米厚的金属薄膜出射而形成光电流。
	\end{physics}
	
	\only<1>{\begin{center}
	\includegraphics[width=0.4\linewidth]{ATR}
	\end{center}}
	
	\only<2>{
	ATR 光阴极具有下面特点:
	\begin{itemize}
	\item 量子效率可以比同材料光滑阴极高 1 $\sim$ 2 个量级
	\item 由于电子出射面光滑,发射度与同材料光滑阴极相当
	\item 激光采用背入射方式激发光电子
	\item 应用于微波电子枪时的微波泄漏的问题还需解决
	\end{itemize}}
\end{frame}

%\begin{frame}[label=current]
%	\frametitle{冷阴极}
%\end{frame}
%
%\begin{frame}[label=current]
%	\frametitle{复合阴极}
%\end{frame}

\end{document}

%\section{Main Part}
%While this text is not shown in the presentation, the section command
%also applies to the presentation.
%We can add a subsection that is only part of the article like this:
%\subsection<article>{Article-Only Section}
%With some more text.
%\begin{frame}
%  This text is part both of the article and of the presentation.
%  \begin{itemize}
%\item This stuff is also shown in both version.
%\item This too.
%  \only<article>{\item This particular item is only part
%      of the article version.}
%  \end{itemize}
%\end{frame}
